\documentclass[11pt]{amsart}
\usepackage[utf8x]{inputenc}
\usepackage{amsthm,amssymb, amsmath}
\usepackage{graphicx}
\usepackage[all]{xy}
\usepackage[dvipsnames]{xcolor}
\usepackage{tikz}

\newtheorem{theorem}{Theorem}[section]
\newtheorem{corollary}[theorem]{Corollary}
\newtheorem{conjecture}[theorem]{Conjecture}
\newtheorem{lemma}[theorem]{Lemma}
\newtheorem{proposition}[theorem]{Proposition}
\newtheorem{assumption}{Assumption}
\newtheorem{definition}[theorem]{Definition}
\newtheorem{remark}[theorem]{Remark}

\newcommand{\matr}[4]{
\left( \begin{array}{cc} #1 & #2 \\ #3 & #4 \end{array} \right)}
\newcommand{\vect}[2]{
\left( \begin{array}{c} #1 \\ #2 \end{array} \right)}
\newcommand{\EE}{\mathcal{E}}
\newcommand{\BB}{\mathcal{B}}
\newcommand{\var}{\textup{Var}}

%opening
\title{Metrics on trees II. Measured laminations and core entropy}
\author{Giulio Tiozzo}
\address{University of Toronto}
\email{tiozzo@math.utoronto.ca}

\date{\today}

\begin{document}
\maketitle





\section{Introduction}

In this paper, we shall introduce a new interpretation of the core entropy for polynomial maps in terms 
of metrics on trees and transverse measures to laminations. 

In the well-known Sullivan dictionary, there is a parallel between the dynamics of Kleinian groups 
and the dynamics of complex polynomials, both seen as dynamical systems on the Riemann sphere.

Measured laminations and measured foliations play a well-known role in Teichm\"uller theory \cite{Th-diffeo}, \cite{FLP}, \cite{Ha}. 

W. Thurston also introduced invariant laminations for quadratic polynomials in \cite{Th}. 
The theory for polynomials of higher degree is still less established \cite{people}. 

However, in Thurston's work on quadratic polynomials there does not appear to be any transverse measure. 
Part of the goal of this paper is to introduce a notion of transverse measure on the lamination associated to a 
complex polynomial.

Further, recall that Thurston's algorithm for constructing a rational map with given combinatorial data is 
obtained by iteration in Teichm\"uller space.
The main result of this part shows convergence of the iterated process to a linearly expanded 
transverse measure on the lamination. 

\begin{theorem}
Let $f : \mathbb{C} \to \mathbb{C}$ be a quadratic polynomial, and let $\mathcal{L}$ be its associated lamination. 
Then: 
\begin{enumerate}
\item
there exists a transverse measure $m$ on $\mathcal{L}$, linearly expanded by the dynamics: 
$$f^\star m = \lambda m.$$
\item 
The derivative of $m$ semiconjugates the dynamics of $f$ on its Hubbard tree to a piecewise linear map.
\item
The number $\lambda$ is the leading eigenvalue of the transfer operator. Moreover, it is related to the core entropy by 
$$h_{top}(f) = \log \lambda.$$
\item
If $h(f) > \frac{1}{2}\log 2$, the metrics $m_n := (f^\star)^n m_0$ converge projectively to $m$.
\end{enumerate}

\end{theorem}

\section{Case II: tree maps}

Recall that a \emph{topological graph} is a topological space given by gluing finitely many segments 
$I_1, \dots, I_r$ along their endpoints. A \emph{topological tree} is a connected, simply connected topological graph.

A continuous function $f : T \to T$ is \emph{folding} at a point $x \in T$ if it is locally modeled by $x \mapsto |x|$; namely, there exist neighborhoods $U$ of $x$, $V$ of $y := f(x)$ and homeomorphisms 
$\varphi_1 : U \to (- \epsilon, \epsilon)$, with $\varphi_1(x) = 0$ and $\varphi_2 : V \to (-\epsilon, \epsilon)$ with $\varphi_2(y) = 0$, such that 
$\varphi_2(f(x)) = |\varphi_1(x)|$ for all $x \in U$.
{\color{blue} Something must be said if $y$ is an endpoint of $T$}

\begin{definition}
A continuous map $f : T \to T$ of a topological tree is \emph{multimodal} if there is a finite set $C \subset T$ such that:  
\begin{itemize}
\item $f$ is a local homeomorphism at every $x \in T \setminus C$;
\item at any point $x \in C$,  $f$ is folding.
\end{itemize}
We say that $f$ is  \emph{unimodal} if $\# C = 1$.
\end{definition}

A fundamental example for our discussion is the following. 

\medskip \noindent \textbf{Example.} Let $f_c(z) := z^2  +c$ be a quadratic polynomial so that its Julia set is connected and locally connected. 
If the Hubbard tree of $f_c$ is a finite topological tree, we say that $f_c$ is \emph{topologically finite}. 
A topologically finite quadratic polynomial acting on its Hubbard tree is an example of a unimodal tree map. 
\medskip

Let $f : T \to T$ be a multimodal map of a tree. An \emph{arc} on $T$ is an unordered pair $[x, y]$ of distinct points of $T$.  
We denote as $\mathcal{A}(T)$ the set of arcs of $T$. 

\begin{definition}
A \emph{metric} on $T$   is a functional
$$m : \mathcal{A}(T) \to \mathbb{R}$$  
which we think of as assigning a \emph{length} to each arc in the tree, 
and satisfies:  
\begin{itemize}
\item[(i)] 
$m$ is \emph{positive}: $m([x,y]) \geq 0$ for any $x,y \in T$;  
\item[(ii)]
$m$ is \emph{additive} on disjoint arcs: if $y$ lies between $x$ and $z$, then $m([x,z]) = m([x,y]) + m([y, z])$.  
\end{itemize}
\end{definition}


Given two points $x, y$ in $T$, we denote as $[x, y]$ the segment (arc) joining $x$ and $y$. 
An arc $[x, y]$ in the tree is \emph{separated} if $x$ and $y$ lie on opposite sides of the critical point.  
Otherwise, it is \emph{non-separated}. 

Given a finite set $x_1, \dots, x_r$ of points of $T$, we define its \emph{convex hull} as $[x_1, \dots, x_r]$ the 
union of all segments $[x_i, x_j]$ with $1 \leq i, j \leq r$.

Let $T \setminus C = I_1 \cup \dots \cup I_d$ the union of disjoint subtrees. 
Note that by definition the restriction of $f$ to any $I_j$ is a homeomorphism onto its image.

For each arc $J \in \mathcal{A}(T)$, we let
$$(f^\star m)(J) := \sum_{k=1}^d m(f (J \cap I_k)).$$

The space of metrics of unit length is $\mathcal{M}^1(T) = \{ m \in \mathcal{M}(T) \ : \ m(T) = 1 \}$.
{\color{blue} Maybe we should define $m$ on any subtree?}

\begin{lemma} \label{L:compact}
The space $\mathcal{M}^1(T)$ is convex and compact. 
\end{lemma}

\begin{proof}
We know that $\mathcal{M}^1(T) \subseteq [0,1]^{\mathcal{A}(T)}$, hence compactness follows from Tychonoff's theorem.
\end{proof}


We define the iteration operator $P: \mathcal{M}^1(T) \to \mathcal{M}^1(T)$ as 
$$P(m) := \frac{f^\star m}{\Vert f^* m \Vert}.$$

\begin{lemma} \label{L:continuous}
The operator $P : \mathcal{M}^1(T) \to \mathcal{M}^1(T)$ is continuous with respect to the weak topology.
\end{lemma}

\begin{proof}
If $m_n \to m$, then for any arc $J$, 
$$f^\star m_n(J) = \sum_i m_n (f(J \cap I_i)) \to \sum_i m(f(J \cap I_i)) = f^\star m(J)$$
and similarly, 
$$\Vert f^\star m_n \Vert = f^\star m_n(T) \to  f^\star m(T) = \Vert f^\star m \Vert  $$
Since $f$ is surjective, note that 
$$f^\star m(T) = \sum_i m(f(I_i)) \geq m(\bigcup_i f(I_i)) \geq m(T) = 1$$
hence $\Vert f^\star m \Vert \geq 1$.
Then 
$$P(m_n)(J) = \frac{f^\star m_n(J)}{\Vert f^\star m_n \Vert} \to \frac{f^\star m(J)}{\Vert f^\star m \Vert} = P(m)(J)$$
which completes the proof.
\end{proof}

\begin{proposition}
There exists at least one unit length metric $m$ on $T$ such that 
$$f^\star m  = \lambda m$$
for some $\lambda \geq 1$.
\end{proposition}

\begin{proof}
By Lemma \ref{L:compact}, 
$\mathcal{M}^1(I)$ is a non-empty, compact, convex subspace of the normed vector space 
$$L^\infty(\mathcal{A}) := \left\{  m : \mathcal{A} \to \mathbb{R} \ : \ \sup_{J \in \mathcal{A}} |m(J)| < + \infty \right\}$$
and $P$ is continuous by Lemma \ref{L:continuous}, hence by the Schauder fixed point theorem, there exists at least one fixed point for the operator $P$, which corresponds to an eigenvector for $f^\star$. 
\end{proof}

\begin{remark}
Note that in this generality, if we do not impose any countable additivity or compactness, 
the eigenvector can be very weird. In particular, consider the map 
$f : [0, 1] \to [0,1]$ defined as 
$f(x) = \frac{x}{1+x}$. 
Then we may have 
$m(J) = 0$ if $J$ does not contain $0$, and 
$m(J) = 1$ if $J$ contains $0$. 
\end{remark}

Let $N$ be the number of ends of $T$.

\begin{proposition}
Let $f : T \to T$ be a critically finite unimodal map, with $h(f) > \frac{1}{2} \log 2$.
Then the transition matrix of $f$ has a unique, real leading eigenvalue of maximum modulus, 
and the eigenvalue is simple. 
\end{proposition}

%\begin{proof}
%Let $J$ be an interval of the partition of $I$. There are two cases:
%\begin{enumerate}
%\item there exists $k \geq 0$ such that $0$ belongs to $f^k(J) \cap f^{k+1}(J) \cap \dots \cap f^{k+N-1}(J)$.
%Then $f^{k+N}(J)$ contains the tree $[f(0), f^2(0), \dots, f^N(0)]$, which is the core.
%We call such an interval \emph{eventually onto}.
%\item if $J$ is not eventually onto, then the critical point never belongs to both $f^k(J)$ and $f^{k+1}(J)$ for any $k$.
%Then $\ell(f^{N k}, J) \leq 2^{(N-1)k}$.
%\end{enumerate}
%Note that the image of a not eventually onto interval is by definition a union of not eventually onto intervals. 
%Thus, if we reorder the basis vectors so that the ones corresponding to non-eventually onto intervals 
%appear first, the transition matrix for $f$ has a block structure
%$$\left( \begin{array}{ll}
%A & B \\ 
%0 & C \\
%\end{array}\right).$$
%By what we have seen in (2), the spectral radius of the matrix $A$ (which represents the restriction of the linear map
%to the span of non eventually onto intervals) is $\leq 2^{\frac{N-1}{N}}$.
%On the other hand, the block $C$ corresponding to the eventually onto intervals 
%is by construction a primitive matrix (i.e. there exists a power of it which is positive)
%hence by the Perron-Frobenius theorem its leading eigenvalue is real and simple, and there are no 
%other eigenvalues of the same modulus. Since by hypothesis the leading eigenvalue is $> 2^{\frac{N-1}{N}}$, 
%the claim follows.
%\end{proof}

%\begin{proof}
%Let $J$ be an interval of the partition of $I$. There are two cases:
%\begin{enumerate}
%\item there exists $k \geq 0$ such that $0$ belongs to $f^k(J) \cap f^{k+1}(J)$. 
%Then $f^{k+1}(J)$ contains the segment $I = [0, f(0)]$.
%We call such an interval \emph{eventually onto}.

%Now, let us note that 
%$$T = \bigcup_{i = 0}^{N-1} f^i(I)$$
%hence 
%$$\ell(f^n, T) \leq   \sum_{i = 0}^{N-1} \ell(f^{n}, f^i(I)) \leq \sum_{i = 0}^{N-1} \ell(f^{n+i}, I)$$
%thus
%$$\lim_{n \to\infty} \ell(f^n, I)^{1/n} = \lim_{n \to \infty} \ell(f^n, T)^{1/n}.$$

%\item if $J$ is not eventually onto, then the critical point never belongs to both $f^k(J)$ and $f^{k+1}(J)$ for any $k$.
%Then $\ell(f^{2 k}, J) \leq 2^{k}$.
%\end{enumerate}

%Note that the image of a not eventually onto interval is by definition a union of not eventually onto intervals. 
%Thus, if we reorder the basis vectors so that the ones corresponding to non-eventually onto intervals 
%appear first, the transition matrix for $f$ has a block structure
%$$\left( \begin{array}{ll}
%A & B \\ 
%0 & C \\
%\end{array}\right).$$
%By what we have seen in (2), the spectral radius of the matrix $A$ (which represents the restriction of the linear map
%to the span of non eventually onto intervals) is $\leq \sqrt{2}$.
%On the other hand, the block $C$ corresponding to the eventually onto intervals 
%is by construction a primitive matrix (i.e. there exists a power of it which is positive)
%hence by the Perron-Frobenius theorem its leading eigenvalue is real and simple, and there are no 
%other eigenvalues of the same modulus. Since by hypothesis the leading eigenvalue is $> \sqrt{2}$, 
%the claim follows.%
%\end{proof}


\begin{theorem}
Let $f$ be a unimodal tree map. Then its \emph{core entropy} $h(f)$ is related to the spectral radius 
of the pullback operator $f^\star$ on the space $\mathcal{M}(I)$, 
i.e.
$$h(f) = \log \lambda.$$
\end{theorem}


\begin{theorem}
Let $f : T \to T$ be a critically finite unimodal tree map,   
with $h(f) > \frac{1}{2} \log 2$.
Then:
\begin{itemize}
 \item
the sequence of metrics $m_n := P^n(m_0)$ \emph{converges uniformly} to a limit $m_\infty$. 
\item
  The limit metric defines a \emph{semiconjugacy} of the dynamics $f : T \to T$ to a piecewise linear tree map of the same entropy.
\end{itemize}
\end{theorem}


\begin{lemma}
Let $N < \infty$ be the number of ends of a quadratic Hubbard tree $T$. Then for any segment $J \subseteq T$ we have either 
\begin{enumerate}
\item
$\ell(f^n(J)) \leq \left( 2^{\frac{N-1}{N}} \right)^n$ for any $n \geq 0$; 
\item
or there exists $n$ such that $f^n(J) = T$. 
\end{enumerate}
\end{lemma}

\begin{proof}
Suppose that there exists $n$ such that $f^n(J) \ni c_k$ for all $0 \leq k \leq N-1$. 
Then $f^{n+N}(J)$ contains $[c_1, c_2, \dots, c_N]$, which is the Hubbard tree. 
Otherwise, in every block of consecutive $N$ iterates, there is at least one iterate 
that does not hit the critical point. 
Hence 
$$\ell(f^N(J)) \leq 2^{\frac{N-1}{N}}$$
and by iteration we get the result. 
\end{proof}

\begin{remark}
Let us note that it is not true that there exists $n$ such that $f^n([c_0, c_1]) = T$. 
As a counterexample, let us consider the Misiurewicz point with external angle $\theta = \frac{9}{56}$. 
Then we compute 
\begin{itemize}
\item $f([c_0, c_1]) = [c_1, c_2]$
\item $f^2([c_0, c_1]) = [c_2, c_3]$
\item $f^3([c_0, c_1]) = [c_1, c_3]$
\item $f^4([c_0, c_1]) = [c_1, c_2]$
\end{itemize}
which then, since $f^4([c_0, c_1]) = f([c_0, c_1])$, repeats without ever covering $T$.
\end{remark}

\section{Case III: laminations}

Hubbard trees are defined for topologically finite polynomials, and in particular their existence requires 
that the Julia set is path connected. In general, it is known that some Julia sets are not locally connected, 
so considering the Hubbard tree as a subset of the filled Julia set may present some difficulties. 

On the other hand, though, for any external angle $\theta \in \mathbb{R}/\mathbb{Z}$, Thurston defines a lamination of the unit disk, 
invariant under the doubling map. We now see how to define a corresponding object to the previously defined 
metric in terms of a \emph{transverse measure} on the lamination. 

\medskip
\subsection*{Laminations}
A \emph{leaf} $L = (\xi, \eta)$ in the Poincar\'e disk $\mathbb{D}$ is a hyperbolic geodesic in $\mathbb{D}$, 
with boundary points $\xi, \eta \in \partial \mathbb{D}$.
A \emph{lamination} $\mathcal{L}$ of the Poincar\'e disk $\mathbb{D}$ is a closed subset which is the union of disjoint leaves.


W. Thurston \cite{Th} constructs, for every angle $\theta \in \mathbb{T}$, an associated lamination $\mathcal{L}_\theta$ on the disc which is invariant by the doubling map.

%[Insert details here]

\medskip
The (two) longest leaves of the lamination are called \emph{major leaves}, and their common image is called \emph{minor leaf} 
and will be denoted by $m$. Moreover, we let $\beta$ denote the leaf $\{0 \}$, which we will take as the root of the lamination (the notation is due to the fact 
that the ray at angle $0$ lands at the $\beta$-fixed point). The dynamics on the lamination is induced by the dynamics of the doubling map 
on the boundary circle. In particular, let us denote $f : \overline{\mathbb{D}} \to \overline{\mathbb{D}}$ to be a continuous function 
on the filled-in disc which extends the doubling map on the boundary $S^1$. Let us denote by $\Delta$ the diameter of the circle which 
connects the boundary points at angles $\theta/2$ and $(\theta+1)/2$. Then we shall also choose $f$ so that it maps homeomorphically 
each connected component of $\mathbb{D}\setminus \Delta$ onto $\mathbb{D}$.

By mapping $(\xi, \eta)$ to the corresponding point in $\mathbb{T} \times \mathbb{T}$, a lamination can be also seen 
as a closed subset of the $2$-torus. 



\subsection{Segments on laminations} 


A leaf $L$ \emph{separates} two other leaves $L_1, L_2$ if $\mathbb{D} \setminus L$ has two connected components, 
one which contains $L_1$ and the other which contains $L_2$.
Let $\mathcal{L}_1$, $\mathcal{L}_2$ be two leaves of a lamination. 
%Then define the \emph{segment} $[\mathcal{L}_1, \mathcal{L}_2]$ as the set of all leaves which separate 
%$\mathcal{L}_1$ and $\mathcal{L}_2$.


\begin{definition}
Let $\mathcal{L}_1$ and $\mathcal{L}_2$ be two distinct leaves. Then we define the \emph{(combinatorial) segment} $[\mathcal{L}_1, \mathcal{L}_2]$
as the set of leaves $\mathcal{L}$ of the lamination which separate $\mathcal{L}_1$ and $\mathcal{L}_2$. 
\end{definition}

Some simple properties of combinatorial segments are the following: 

\begin{itemize}
\item[(a)]
if $\mathcal{L} \in [\mathcal{L}_1, \mathcal{L}_2]$, then $[\mathcal{L}, \mathcal{L}_1] \subseteq [\mathcal{L}_1, \mathcal{L}_2]$;
\item[(b)]
for any choice of leaves $\mathcal{L}_1, \mathcal{L}_2, \mathcal{L}_3$ we have 
$$[\mathcal{L}_1, \mathcal{L}_2] \subseteq [\mathcal{L}_3, \mathcal{L}_1] \cup [\mathcal{L}_3, \mathcal{L}_2];$$
\item[(c)]
the image of $[\mathcal{L}_1, \mathcal{L}_2]$ equals: 
$$f([\mathcal{L}_1, \mathcal{L}_2])  = 
\left\{ \begin{array}{ll} 
\lbrack f(\mathcal{L}_1), f(\mathcal{L}_2) \rbrack &  \textup{if }\Delta\textup{ does not separate }\mathcal{L}_1\textup{ and }\mathcal{L}_2, \\
\lbrack f(\mathcal{L}_1), m \rbrack \cup \lbrack f(\mathcal{L}_2), m \rbrack & \textup{if }\Delta\textup{ separates }\mathcal{L}_1\textup{ and }\mathcal{L}_2. 
\end{array}
\right.$$
\item[(d)]
in any case, for any leaves $\mathcal{L}_1, \mathcal{L}_1$ we have 
$$[f(\mathcal{L}_1), f(\mathcal{L}_2)]  \subseteq f([\mathcal{L}_1, \mathcal{L}_2]) \subseteq \lbrack f(\mathcal{L}_1), m \rbrack \cup \lbrack f(\mathcal{L}_2), m \rbrack.$$
\end{itemize}

Another important notion is convexity. 

\begin{definition}
We say that a set $S$ of leaves is \emph{combinatorially convex} if whenever $\mathcal{L}_1$ and $\mathcal{L}_2$ 
belong to $S$, then the whole set $[\mathcal{L}_1, \mathcal{L}_2]$ is contained in $S$.
Given a set of leaves $\mathcal{L}'$, we define the \emph{convex hull} $C(\mathcal{L}')$ as the set of leaves $L$ such that 
there exist $L_1, L_2$ in $\mathcal{L}'$ and $L$ separates $L_1$ and $L_2$. 
\end{definition}






%Let $\mathcal{L}$ be a lamination. Given two leaves $L_1, L_2$ of $\mathcal{L}$, we define the 
%\emph{combinatorial segment} $[L_1, L_2]$ as the set of leaves $L$ of $\mathcal{L}$ which separates $L_1$ and $L_2$. 




\subsection{Combinatorial Hubbard trees}

Given a minor leaf $\ell$, we define the \emph{postcritical lamination} $P(\ell)$ as the union of the 
set of leaves 
$$P(\ell) := \{ f^n(\ell), n \geq 0 \}.$$
Moreover, the \emph{combinatorial Hubbard tree}  $\mathcal{H}$ is the convex hull of the postcritical lamination. 


We now define 
$$H_n := \bigcup_{0 \leq i \leq n} [\beta, f^i(m)]$$
and
$$H := \bigcup_{n \in \mathbb{N}} H_n.$$

We call the set $H$ the \emph{combinatorial Hubbard tree} of $f$, as it is a combinatorial version of the (extended) Hubbard tree.

%other definition (not equivalent!) $$H_n := \bigcup_{0 \leq i < j \leq n} [f^i(m), f^j(m)]$$
\begin{lemma}[\cite{ET}] \label{L:invariant}
The combinatorial Hubbard tree $H$ has the following properties.
\begin{enumerate}
\item The set $H$ is the smallest combinatorially convex set of leaves which contains $\beta$, $m$ and is forward invariant.
%$f(H_{N}) \subseteq H_{N}$ and 
\item 
Let $N \geq 0$ be an integer such that $f^{N+1}(m) \in H_N$. Then we have 
$H = H_{N}$.
\end{enumerate}
\end{lemma}

Note that $N + 2$ coincides with the number of ends of the extended Hubbard tree.

A lamination induces an equivalence relation on $\overline{D}$, by identifying every leaf and every polygon to a point. 

\begin{lemma}
The quotient $T := \mathcal{H}/\sim$ is a tree. Moreover, the doubling map induces 
a unimodal tree map $f : T \to T$.
\end{lemma}

\begin{lemma}
If the minor leaf is not degenerate, then the Hubbard tree has a finite number of ends. 
\end{lemma}

\begin{proof}
Since the doubling map doubles the length of small leaves, there must be $k > n > p$ so that 
$f^k(m)$ separates $f^n(m)$ and $f^p(m)$. 
\end{proof}



\begin{lemma}
Let $m_1 < m_2$, and $H_1, H_2$ be the corresponding combinatorial Hubbard trees. Then 
$$H_1 \subseteq H_2.$$
As a corollary, 
$$\#Ends(T_1) \leq \#Ends(T_2).$$
\end{lemma}

%\begin{proof}
%By definition, $m_1 < m_2$ means that $m_1 \in [\beta, m_2]$. Thus, $m_1 \in H_2$ and since $H_2$ is forward invariant we have $f^i(m_1) \in H_2$ 
%%for any $i \geq 0$.  Since also $\beta \in H_2$ and $H_2$ is combinatorially convex, then $[\beta, f^i(m_1)] \subseteq H_2$ for any $i \geq 0$, thus 
%$H_1 \subseteq H_2$ as required. For the corollary, note that since the trees are dual to the laminations, $T_1 \subseteq T_2$, 
%and in general a connected subtree of a tree cannot have more ends than the ambient tree. 
%\end{proof}

%\begin{lemma} \label{L:disjoint-intervals}
%Let $(\theta^-, \theta^+)$ be a ray pair, and $\theta_0$ its pseudocenter, with $\Vert \theta_0 \Vert = q$. Then 
%$$\#Ends(T_{\theta^+}) = \#Ends(T_{\theta_0})$$
%%if and only if the arcs $I_k = (D^k(\theta^-), D^k(\theta^+))$ for $k = 0, \dots, q - 1$ are disjoint.
%\end{lemma}

% \begin{lemma} \label{l:pseudoiterate}
%Let $[\alpha, \beta]$ be an embedded arc in $S^1$ which does not contain $0$, and $\theta_0$ its pseudocenter, with $q = \Vert \theta_0 \Vert$.
%Then for all $0 \leq k \leq q-1$, the arc $[D^k(\alpha), D^k(\beta)]$ is embedded and does not contain $0$. 
%\end{lemma}


\subsection{Core entropy}


Definition of core entropy in terms of laminations. 

\begin{definition}
Let $\ell$ be the minor leaf, and $\mathcal{H}$ the combinatorial Hubbard tree. 
Then the core entropy of $\mathcal{L}$ is 
$$h(\mathcal{L}) := \lim_{n \to \infty} \frac{1}{n} \log \# \{ L \in \mathcal{H} \ : \ f^n(L) = \ell \}$$
\end{definition}


\medskip
\noindent \textbf{Question.} Does this definition of core entropy coincide with the one given by Thurston's algorithm?


\medskip
For tree maps, a semiconjugacy to a piecewise linear models for tree maps is constructed in Baillif-DeCarvalho (Theorem 4.3). 


\subsection{Transverse measures}

%\begin{definition}
%A transverse measure on the lamination $\mathcal{L}$ is a functional 
%$m$ on the set of segments, which is non-negative and countably additive. 
%\end{definition}


Let $\mathcal{L}$ be a quadratic lamination, with minor leaf $L_{\min}$. 
Let $\mathcal{A}(L)$ be the set of segments of $\mathcal{L}$. 

\begin{definition}
A \emph{transverse measure} on $\mathcal{L}$   is a functional
$$m : \mathcal{A}(L) \to \mathbb{R}$$  
which satisfies:  
\begin{itemize}
\item[(i)] 
$m$ is \emph{positive}: $m([L_1,L_2]) \geq 0$ for any $L_1, L_2 \in \mathcal{L}$;  
\item[(ii)]
$m$ is \emph{additive} on disjoint arcs: 
$$m([L_1, L_3]) = m([L_1, L_2]) + m([L_2, L_3])$$
if $L_2$ separates $L_1, L_3$.
\end{itemize}
\end{definition}

The value $m(L_1, L_2)$ is thought of as the measure of the segment $[L_1, L_2]$ joining $L_1$ to $L_2$ in the dual tree. 
%This is also the set of leaves of $\mathcal{L}$ separating $L_1$ and $L_2$.). 
We denote the space of transverse measures on $\mathcal{L}$ as $\mathcal{M}(\mathcal{L})$.

\subsection{The pullback operator}


%We define a pullback operator $f^\star : \mathcal{M}(\mathcal{L}) \to \mathcal{M}(\mathcal{L})$ 
%on the set of all transverse measures as 
%$$f^\star m([L_1, L_2]) := \sum m([f(L_1), f(\Delta)])$$
%where $\Delta = \left( \frac{\theta}{2}, \frac{\theta}{2} + \frac{1}{2} \right)$ is the critical diagonal (or better to choose the major leaf?).

We now define the pullback operator $f^* : \mathcal{M}(\mathcal{L}) \to \mathcal{M}(\mathcal{L})$ 
on the set of metrics as: 
$$f^*m(L_1, L_2) = m(f(L_1), f(L_2))$$
if $L_1, L_2$ are not separated, while 
$$f^*m(L_1, L_2) = m(f(L_1), L_{\min}) + m(L_{\min}, f(L_2))$$
if $L_1, L_2$ are separated.




\begin{definition}
We call a metric $\lambda$ \emph{linearly expanded} by $f$ if there exists $\lambda \in \mathbb{R}$ such that 
$$f^\star m = \lambda m.$$
\end{definition}

\noindent Linearly expanded, non-zero metrics are fixed points of the operator
$$P(m) := \frac{f^\star m}{\Vert f^\star m \Vert}.$$


\begin{lemma} \label{L:continuous}
The operator $P : \mathcal{M}^1(T) \to \mathcal{M}^1(T)$ is continuous with respect to the weak topology.
\end{lemma}

\begin{proof}
If $m_n \to m$, then for any arc $J$, 
$$f^\star m_n(J) = \sum_i m_n (f(J \cap I_i)) \to \sum_i m(f(J \cap I_i)) = f^\star m(J)$$
and similarly, 
$$\Vert f^\star m_n \Vert = f^\star m_n(T) \to  f^\star m(T) = \Vert f^\star m \Vert  $$
Since $f$ is surjective, note that 
$$f^\star m(T) = \sum_i m(f(I_i)) \geq m(\bigcup_i f(I_i)) \geq m(T) = 1$$
hence $\Vert f^\star m \Vert \geq 1$.
Then 
$$P(m_n)(J) = \frac{f^\star m_n(J)}{\Vert f^\star m_n \Vert} \to \frac{f^\star m(J)}{\Vert f^\star m \Vert} = P(m)(J)$$
which completes the proof.
\end{proof}

\begin{proposition}
There exists at least one unit length transverse measure $m$ such that 
$$f^\star m  = \lambda m$$
for some $\lambda \geq 1$.
\end{proposition}

\begin{proof}
By Lemma \ref{L:compact}, 
$\mathcal{M}^1(I)$ is a non-empty, compact, convex subspace of the normed vector space 
$$L^\infty(\mathcal{A}) := \left\{  m : \mathcal{A} \to \mathbb{R} \ : \ \sup_{J \in \mathcal{A}} |m(J)| < + \infty \right\}$$
and $P$ is continuous by Lemma \ref{L:continuous}, hence by the Schauder fixed point theorem, there exists at least one fixed point for the operator $P$, which corresponds to an eigenvector for $f^\star$. 
\end{proof}



\begin{lemma}
A linearly expanded metric $m$ defines a semiconjugacy of $f$ to a piecewise linear map: indeed, if one defines 
$$h(x) := m([0, x])$$
then one has 
$$h \circ f = g \circ h$$
where $g$ is the piecewise linear map with slope $\lambda$ and 
critical points $h(c_i)$, where $c_i$ is a critical point for $f$.
\end{lemma}


%\begin{theorem}
%Let $\mathcal{L}$ be a quadratic lamination. Then there exists a transverse measure $m$ on $\mathcal{L}$ and a constant 
%$\lambda \geq 1$ such that 
%$$f^* m = \lambda m.$$
%Then, if the lamination $\mathcal{L}$ is associated to a quadratic polynomial $f_c$, then its core entropy is
%$$h(f_c) = \log \lambda.$$
%Finally, integrating with respect to $m$ gives a piecewise linear model of $f_c$ (restricted to its Hubbard tree)
%whose derivative has modulus $\lambda$.
%\end{theorem}



\section{The transfer operator}

\subsection{Functions of bounded variation}

Let $L = \{ L_1,\dots, L_k \}$ be a finite set of leaves of the lamination. 
Two leaves are \emph{adjacent} in $L$ if they are not separated by another element of $L$. 
A set $L$ is \emph{non-overlapping} if, whenever $\{L_1, L_2\}$ and $\{L_3, L_4\}$ are distinct pairs of
adjacent leaves, the (open) intervals $(L_1, L_2)$ and $(L_3, L_4)$ are disjoint. 

Let $L$ be a non-overlapping set, and $f : \Lambda(\mathcal{L}) \to \mathbb{R}$ be a function. We define its \emph{total variation} over $L$ as
%{\color{blue} (What if $f : \mathcal{L} \to \mathcal{L}$?)}
$$\textup{Var}(f; L) := \sum_{\{L_i, L_j\} \in \textup{adj}(L)} |f(L_i) - f(L_j)|$$
where the index runs over the set $\textup{adj}(L)$ of adjacent leaves. 
%{\color{red} Definition is not monotonic is mesh of $L$}

\begin{definition}
A function $f : \textup{Leaves}(\mathcal{L}) \to \mathbb{R}$ is of \emph{bounded variation} if 
$$\textup{Var}(f) := \sup_{L} \textup{Var}(f; L) < + \infty$$
where the $\sup$ runs over all finite non-overlapping subsets of $\mathcal{L}$. 
\end{definition}

\subsection{Integration}

Let $\Lambda(\mathcal{L})$ be the space of leaves of the lamination $\mathcal{L}$. 
Let $f : \Lambda(\mathcal{L}) \to \mathbb{R}$ be a function, and let $m$ be a transverse measure to the lamination $\mathcal{L}$. 

Let $\mathcal{P} = \{L_1, \dots, L_k\}$ be a finite, non-overlapping set of leaves of $\mathcal{L}$.
Let the \emph{mesh} of $\mathcal{P}$ be 
$$|\mathcal{P}| := \sup_{L_i, L_j \in \textup{adj(L)}} d(L_i, L_{j})$$
where the distance can be taken e.g. as points in $S^1 \times S^1 /\sim$. % (or as subsets of $\overline{\mathbb{D}}$).

The space $\mathcal{S}$ of \emph{simple functions} is the space of all finite linear combinations 
of the characteristic functions $\chi_J$ of all subintervals $J \subset I$. 
A metric $m$ defines a linear operator $\mathbb{E}_m : \mathcal{S} \to \mathbb{R}$ by
$$\mathbb{E}_m(\varphi) := \sum_J a_J m(J)$$
if $\varphi = \sum_J a_J \chi_J$. We also write $\int \varphi \ dm$ instead of $\mathbb{E}_m(\varphi)$. 
Recall the well-known

\begin{lemma}  \label{L:dense}
The space $\mathcal{S}$ %an of the characteristic functions of intervals 
is dense in the space BV.
\end{lemma}

By the Lemma, we can extend the operator $\mathbb{E}_m$ to $\mathbb{E}_m : BV \to \mathbb{R}$, 
defining 
$$\int \varphi(x) \ dm(x) := \sup \{ \mathbb{E}_m(\psi) \ : \ \psi \in \mathcal{S}, \psi \leq \varphi \}.$$
%$$\lim_{|\mathcal{P}| \to 0} \sum_{i = 1}^{\# \mathcal{P}} \varphi(x_i) m([x_i, x_{i+1}])$$
Hence we obtain
pairing $\langle \cdot, \cdot \rangle : BV \times \mathcal{M} \to \mathbb{R}$,
$$\langle \varphi, m \rangle := \int \varphi \ dm.$$



\subsection{The transfer operator}

Consider a lamination $\mathcal{L}$ as a subset of $\mathbb{T} \times \mathbb{T}$. 
The map $\sigma(x,y) := (2 x, 2 y) \mod 1$ satisfies $\sigma(\mathcal{L}) \subseteq \mathcal{L}$ and 
$\sigma(\mathcal{H}) \subseteq \mathcal{H}$. 

Moreover, the critical diameter $\Delta$ separates $\mathcal{L}$ in two parts, $\mathcal{L} = \mathcal{L}_1 \cup \mathcal{L}_2$. 
Let $\sigma_i^{-1} : \mathcal{L}_i \to \mathcal{L}$ be the local inverses of $\sigma$. 

We define the transfer operator $\mathcal{L} : BV \to BV$ as 
$$\mathcal{L}\varphi (x) := \sum_{i = 1}^d \varphi(\sigma_i^{-1} x)$$



\begin{lemma}
For any interval $I$ and any metric $m$, we have 
$$\langle \mathcal{L} \chi_I, m \rangle = \langle \chi_I, f^\star m \rangle$$
\end{lemma}

%\begin{proof}
%Then we have 
%\begin{align*}
%\int \mathcal{L} \chi_I \ dm & = \sum_{i = 1}^d \int \chi_I(f_i^{-1} x) \ dm(x)\\
%& = \sum_{i = 1}^d \int \chi_{f(I \cap I_i)}( x) \ dm(x) \\
%& = \sum_{i = 1}^d m(f(I \cap I_i)) \\
%& = f^\star m(I).
%\end{align*}
%\end{proof}

%In fact, 
As a corollary, using Lemma \ref{L:dense}, 
%if we consider the pairing $\langle \cdot, \cdot \rangle : BV \times \mathcal{M} \to \mathbb{R}$,
%$$\langle \varphi, m \rangle := \int \varphi \ dm$$
we obtain the duality relation % that $\mathcal{L}$ and $P$ are dual, in the sense that 
$$\langle \mathcal{L} \varphi, m \rangle = \langle \varphi, f^\star m \rangle$$
for any $\varphi \in BV$, $m \in \mathcal{M}$. 
Recall also that 
$$\textup{Var}_J (f^n) = \Vert \mathcal{L}^n \chi_J \Vert_{L^1}.$$



\begin{definition}
A subset $J \subseteq \mathcal{H}$ is a \emph{monotone set} if the restriction $\sigma\vert_{\mathcal{J}} : \mathcal{J} \to \sigma(\mathcal{J})$ 
is a homeomorphism onto its image. 

Further, a subset $J \subseteq \mathcal{H}$ is \emph{convex} if, for any $x, y \in \mathcal{J}$, the segment $[x, y] \cap \mathcal{J}$ is contained in $\mathcal{J}$. 
\end{definition}

Given a convex set $J$, we denote as $\mathcal{E}(J)$ its number of ends, that is the maximal 
cardinality of a set of leaves of $J$ such that no three of them are nested. 
We have the properties: 
\begin{enumerate}
\item if $J$ is a convex set, 
$$\var_J(\varphi_1 \varphi_2) \leq \var_J(\varphi_1) \sup_J |\varphi_2| + \var_J(\varphi_2) \sup_J |\varphi_1|$$
\item if $J \subseteq K$ are convex sets, 
$$\var_K (\varphi \cdot \chi_J) \leq \var_J (\varphi) + \mathcal{E}(K) \sup_J |\varphi|$$
%(in fact, $\mathcal{E}(J)$ is enough)
\item if $J$ is a monotone set,
$$\textup{Var}_J (\varphi \circ \sigma) = \textup{Var}_{\sigma(J)}(\varphi)$$
%and $E(K)$ is the number of ends of $K$.
\end{enumerate}

We consider as $\mathcal{Z}$ the partition of $\mathcal{H}$ in the two components of $\mathcal{H} \setminus \Delta$. 
Both such components are monotone sets. For each $m$, we denote as 
$$\mathcal{Z}_m := \{ \eta_0 \cap f^{-1}(\eta_1) \cap \dots \cap f^{-(m-1)}(\eta_{m-1}) \ : \ \eta_i \in \mathcal{Z} \}$$


\subsection{Spectral decomposition}


\begin{lemma}
If $J \subseteq \eta \in \mathcal{Z}_m$ is a convex set and $\varphi : J \to \mathbb{C}$, 
$$\Vert \mathcal{L}^m(\varphi \cdot \chi_J) \Vert_{BV} \leq \var_J( \varphi) + (\mathcal{E}(I) + 1) \sup_J |\varphi|$$
\end{lemma}

\begin{proof}
Since $J$ is a subset of an element of $\mathcal{Z}_m$, 
$$ \mathcal{L}^m(\varphi \cdot \chi_J)  = (\varphi \cdot \chi_J) \circ T_\eta^{-m}$$
hence
$$ \var_I \mathcal{L}^m(\varphi \cdot \chi_J)  \leq \var_J (\varphi) + \mathcal{E}(I) \sup_J |\varphi| $$
while 
$$\Vert \mathcal{L}^m(\varphi \cdot \chi_J) \Vert_\infty \leq \sup_J |\varphi|.$$
\end{proof}


Now, for each $\eta$, pick a point $x_\eta \in \eta$. 
consider for any $m$ the operator 
$$F_m(f) =  \sum_{\eta \in \mathcal{Z}_m} f(x_\eta) \mathcal{L}^m \chi_\eta$$ 
which has finite range. 




\begin{lemma} \label{L:almost-compact}
We have 
$$\sum_{\eta \in \mathcal{Z}_m} \Vert \mathcal{L}^m \chi_\eta \cdot f(x_\eta) - \mathcal{L}^m(f \cdot \chi_\eta)\Vert_{BV} \leq C \cdot \var(f)$$
\end{lemma}

\begin{proof}
\begin{align*}
\sum_{\eta \in \mathcal{Z}_m} \Vert \mathcal{L}^m \chi_\eta \cdot f(x_\eta) - \mathcal{L}^m(f \cdot \chi_\eta)\Vert_{BV} 
& \leq \sum_{\eta \in \mathcal{Z}_m} \Vert \mathcal{L}^m \chi_\eta \cdot (f(x_\eta) - f)\Vert_{BV} \\
& \leq \sum_{\eta \in \mathcal{Z}_m}
\var_\eta (f(x_\eta) - f) + (\mathcal{E}(I) + 1) \sup_\eta |f(x_\eta) - f| \\
& \leq \sum_{\eta \in \mathcal{Z}_m}
(\mathcal{E}(I) + 2) \var_\eta (f(x_\eta) - f) \\
&  \leq (\mathcal{E}(I) + 2)  \sum_{\eta \in \mathcal{Z}_m}
 \var_\eta f \leq (\mathcal{E}(I) + 2)  \var_I f
 \end{align*}
\end{proof}


Now, from Lemma \ref{L:almost-compact}
\begin{align*}
\Vert \mathcal{L}^m(f) - F_m(f) \Vert_{BV} & = \Vert  \sum_{\eta \in \mathcal{Z}_m} \mathcal{L}^m(f \chi_\eta) -  f(x_\eta) \mathcal{L}^m \chi_\eta \Vert_{BV}\\
& \leq  \sum_{\eta \in \mathcal{Z}_m}  \Vert\mathcal{L}^m(f \chi_\eta) -  f(x_\eta) \mathcal{L}^m \chi_\eta  \Vert_{BV} \\
& \leq C \cdot \var(f)
\end{align*}
Since $F_m$ has finite range, hence it is compact, it follows that $\mathcal{L}$ is quasicompact, 
hence, e.g. from Dunford-Schwartz (VIII.8.2) it follows that $\mathcal{L}$ has the following 
spectral decomposition. 

{\color{blue} Still to prove it has cyclic spectrum}

\begin{theorem}
Let $\mathcal{L}$ be a quadratic lamination, and suppose that the associated combinatorial Hubbard tree $\mathcal{H}$ 
has a finite number of ends. 

Suppose $f : I \to I$ is a piecewise monotone map with $h_{top}(f) > 0$.
Let $\mathcal{L} : BV \to BV$ be the transfer operator. Then: 
\begin{enumerate}
\item the spectral radius of $\mathcal{L}$ equals $\lambda = e^{h_{top}(f)}$; 
\item there exists a spectral decomposition 
$$\mathcal{L} = \sum_{i = 0}^r \lambda_i P_i + P_{res}$$
where $\lambda = \lambda_0, \lambda_1, \dots, \lambda_r$ are the eigenvalues of modulus $> 1$, each $P_i$ is a spectral projector of finite rank, and $P_{res}$ has spectral radius $\leq 1$. 
\end{enumerate}
\end{theorem}

%\begin{proof}
%This follows immediately from \cite[Theorem 1]{Baladi-Keller}, with potential $g \equiv 1$. 
%The essential spectral radius of the transfer operator is (\cite[Theorem 1]{Ke})
%$$\rho_{ess} := \lim_{n \to \infty} \left( |g \cdot g \circ f \cdot g \circ f^2 \dots \cdot g \circ f^{n-1}| \right)^{1/n} = 1.$$
%Moreover, by 
%\cite[Theorem 3]{Baladi-Keller} applied to $g \equiv 1$, one has: 
%$$h_{top}(f) = P(f, 0) = \log \rho(\mathcal{L}).$$
%\end{proof}


\subsection{Spectral radius and core entropy}

In \cite{Ti}, we construct an infinite graph $W_\theta$ in order to compute the core entropy. 
More precisely, the vertex set of the graph is the set 
$$\Sigma := \{ (i, j) \ : \ 0 \leq i < j \}$$
of pairs, called a \emph{wedge}. 

\begin{theorem}[\cite{Ti}]
The core entropy of $f_\theta$ satisfies 
$$h(f_\theta) = \lim_{n \to \infty} \frac{1}{n} \log \# \{ \textup{closed paths in }W_\theta \textup{ of length }n \}$$
\end{theorem}

Recall that the spectral radius of $\mathcal{L}$ is given by 
$$\rho(\mathcal{L}) = \lim_n \Vert \mathcal{L}^n 1 \Vert_{BV}^{1/n}$$
Since 
$$\mathcal{L}^n 1 = \sum_{\eta \in \mathcal{Z}_n} \chi_{f^n(I_\eta)}\circ f_\eta^{-n}$$
we have 
$$ \Vert \mathcal{L}^n 1 \Vert \leq \mathcal{E}(I) \# \mathcal{Z}_n$$
and also 
$$\mathcal{L}^n 1(x) = \#\{ f^{-n}(x) \}$$

We need to show 
\begin{equation} \label{E:count-Zn}
h(f_\theta) = \lim_{n \to \infty} \frac{1}{n} \log \# \mathcal{Z}_n
\end{equation}

From the above equation, proceeding as in the proof of \cite[Theorem 3]{Baladi-Keller}, we obtain 
$$h(f_\theta) = \log \rho(\mathcal{L}).$$

\begin{lemma}
We have 
$$h(f_\theta) = \log \rho(A_\theta)$$
where $A_\theta : \ell^1(W_\theta) \to \ell^1(W_\theta)$ is the adjacency operator of the infinite graph. 
\end{lemma}

We now prove that the transfer operator and the adjacency operator have equal spectral radius. 

\begin{theorem}
We have 
$$\rho(A_\theta) = \rho(\mathcal{L})$$
\end{theorem}

We consider the map $i : \ell^1(W_\theta) \to BV(\mathcal{L})$ given by 
$$i(u) := \sum_{i < j} u(i, j) \chi_{[\theta_i, \theta_j]}$$
where $\theta_i := 2^{i-1} \theta \mod 1$. 
Note that $i$ is a continuous (bounded) operator.


Since every edge of the tree has two sides, if the tree is topologically finite there exists $u \in \ell^1$ such that 
$i(u) = 1$.
Then 
$$\Vert \mathcal{L}^n 1 \Vert_{BV} = \Vert \mathcal{L}^n i (u) \Vert_{BV} = \Vert i A^n(u) \Vert_{BV} \leq \Vert A^n(u) \Vert_{\ell^1}
\leq \Vert A^n \Vert_{\ell^1} \cdot \Vert u \Vert_{\ell^1}$$
hence 
$$\rho(\mathcal{L}) = \lim_n \Vert \mathcal{L}^n 1 \Vert_{BV}^{1/n} \leq \lim_n  \Vert A^n \Vert_{\ell^1}^{1/n} = \rho(A_\theta)$$

\begin{thebibliography}{99}

\bibitem{Baladi-book}
V. Baladi, \emph{Positive transfer operators and decay of correlations}, 
World Scientific, Singapore, 2000.

\bibitem{Baladi-Keller}
V. Baladi, G. Keller, 
\emph{Zeta functions and transfer operators for piecewise monotone transformations}
Commun. Math. Phys. 127 (1990), 459--477.


\bibitem{BMS}
A. Bonifant, J. Milnor and S. Sutherland, 
\emph{The W. Thurston Algorithm Applied to Real Polynomial Maps}, 
preprint \texttt{arXiv:2005.07800}.

\bibitem{BH}
S. Boyd and C. Henriksen, \emph{The Medusa algorithm for polynomial matings}, 
Conform. Geom. Dyn. 16 (2012), 161--183.

\bibitem{Ch}
D. Chandler, \emph{Extrema of a Polynomial}, 
Amer. Math. Monthly 64 (1957), no. 9, 679--680.

\bibitem{DH}
A. Douady and J. Hubbard, \emph{A proof of Thurston's topological characterization of rational functions}, 
Acta Math. 17 (1993), 263--297.

\bibitem{ET}
A. Epstein and G. Tiozzo, \emph{Generalizations of Douady's magic formula}, 
Ergodic Theory Dynam. Systems (2021), to appear. 

\bibitem{FLP}
A. Fathi, F. Laudenbach and V. Po\'enaru, \emph{Travaux de Thurston sur les surfaces},
Ast\'erisque, no. 66-67 (1979).

\bibitem{Ha}
A. Hatcher, \emph{Measured lamination spaces for surfaces, from the topological viewpoint},
Topology and its Applications, Volume 30 (1988), Issue 1, 63--88.

\bibitem{HK}
F. Hofbauer, G. Keller, 
\emph{Zeta-functions and transfer-operators for piecewise linear transformations}, 
J. Reine Angew. Math. 352 (1984), 100--113.


\bibitem{Mil} 
J. Milnor, \emph{Thurston's algorithm without critical finiteness}, in \emph{Linear and
Complex Analysis Problem Book 3, Part 2}, Havin and Nikolskii editors, Lecture
Notes in Math no. 1474, pp. 434-436, Springer, 1994.

\bibitem{Mil-slides}
J. Milnor, \emph{Remarks on Piecewise Monotone Maps}, 2015, 
available at \texttt{https://www.math.stonybrook.edu/~jack/BREMEN/pm-print.pdf}.

\bibitem{MiTr}
J. Milnor and C. Tresser,
\emph{On Entropy and Monotonicity for Real Cubic Maps}, 
Comm. Math. Phys. 209 (2000), 123--178. With an Appendix by A. Douady and P. Sentenac.

\bibitem{MS}
M. Misiurewicz and W. Szlenk, 
\emph{Entropy of piecewise monotone mappings}, 
Studia Math. 67 (1980), 45--63.

\bibitem{Th-diffeo}
W. Thurston, \emph{On the geometry and dynamics of diffeomorphisms of surfaces}, 
Bull. Amer. Math. Soc. (N.S.) 19(2): 417-431 (October 1988).

\bibitem{Th}
W. Thurston, \emph{On the Geometry and Dynamics of Iterated Rational Maps}, 
in D. Schleicher, N. Selinger, editors, ``Complex dynamics'', 3--137, A K Peters, Wellesley, MA, 2009. 

\bibitem{people}
W. Thurston, H. Baik, Y. Gao, J. Hubbard, K. Lindsey, L. Tan, D. Thurston,
\textit{Degree-d invariant laminations}, to appear in \emph{What's next? The mathematical legacy of Bill Thurston}, Princeton University Press.

\bibitem{Ti}
G. Tiozzo, \emph{Continuity of core entropy of quadratic polynomials},
Invent. Math. (2016).

\bibitem{Wi}
M. Wilkerson, \emph{Thurston's algorithm and rational maps from quadratic polynomial matings},
Discrete Contin. Dyn. Syst. Ser. S 12 (2019), no. 8, 2403--2433.


\end{thebibliography}

\end{document}



\section{Entropy determines the lamination}

\begin{theorem}[Entropy determines the lamination]
Suppose that $h(\theta_1) = h(\theta_2)$ and $h(\theta) > h(\theta_1)$ for any $\theta \in (\theta_1, \theta_2)$. 
Then $R_M(\theta_1)$ and $R_M(\theta_2)$ land together. 
\end{theorem}

\begin{proof}
Let $\theta_3 \in (\theta_1, \theta_2)$. We claim that each element in the same equivalence class as $\theta_3$ belongs to $[\theta_1, \theta_2]$. Indeed, if not that there is $\theta_4$ with $\theta_3 \sim \theta_4$ and the leaf $(\theta_3, \theta_4)$ separates $0$ from either $\theta_1$ or $\theta_2$. Then $h(\theta_4) = h(\theta_3) > h(\theta_1) = h(\theta_2)$ by hypothesis; however, we also have, since $(\theta_3, \theta_4)$ separates $0$ from $\theta_i$ for some $i = 1, 2$, that $h(\theta_3) \leq h(\theta_i)$, which is a contradiction.

Now, by approximating the leaf $(\theta_1, \theta_2)$ by finite chains of leaves lying in the interval $(\theta_1, \theta_2)$, we get that $\theta_1 \sim_M \theta_2$.
\end{proof}

Note that the leaves of QML we find this way are either Feigenbaum parameters at the boundary of the main molecule, or tips of small Mandelbrot sets with roots of positive entropy.

\begin{lemma}
If $h(\theta_1) = h(\theta_2)$ and $h(\theta) \geq h(\theta_1)$ for all $\theta \in (\theta_1, \theta_2)$, then every $\theta_3$ which lies in $(\theta_1, \theta_2)$ and has a companion outside $(\theta_1, \theta_2)$ must satisfy $h(\theta_3) = h(\theta_1)$.
\end{lemma}

\begin{remark}
The same is NOT true if you just require $h(\theta) \geq h(\theta_1)$ for all $\theta \in (\theta_1, \theta_2)$. Indeed, for instance 
$\theta_2$ could be the doubling of $\theta_1$. In a way, that should be the only case.
\end{remark}

\begin{theorem}[Entropy determines the lamination II]
Suppose that $\theta_1 <_M \theta_2$. If $h(\theta_2) = 0$, then both $\theta_1$ and $\theta_2$ belong to the main molecule. 
Then, either $\theta_1, \theta_2$ belong to the same small copy of the Mandelbrot set with root of positive entropy, or $h(\theta_1) < h(\theta_2)$. 
\end{theorem}

\begin{remark}
Suppose $\theta_1$ and $\theta_2$ have the same period and $\theta_1 <_M \theta_2$, and $h(\theta_1) = h(\theta_2)$. 
Then it is NOT true that $\theta_1 \sim_M \theta_2$. Indeed, you can take the airplane tuned with the airbus and the airplane tuned with the double basilica. They both have period $12$, they are nested, and they have the same entropy as they both lie in the airplane small Mandelbrot set. 
\end{remark}

However, we can define the ``multientropy" as the sequence $(h_1, h_2, \dots, h_k)$ of entropies of all the small Hubbard trees 
in the small Julia sets. This should give an invariant, namely: 

\textbf{Conjecture.} 
If two parameters $\theta_1$, $\theta_2$ have the same multientropies and are nested, then $\theta_1 \sim_M \theta_2$.

\begin{theorem}[Local H\"older exponent for the entropy]
For any $\theta \in \mathbb{R}/\mathbb{Z}$, the local H\"older exponent of the core entropy satisfies
$$\alpha(h, \theta) = \frac{h(\theta)}{\log 2}.$$
\end{theorem}

Recall that 
$$\alpha(f, x) := \liminf_{y, z \to x} \frac{\log |f(y) - f(z)| }{\log|y-z|}$$

\subsection{Transverse measures}

Let $\theta \in \mathbb{R}/\mathbb{Z}$ and $\mathcal{L}_\theta$ be the corresponding quadratic lamination. 
Then there exists transverse measure $m_\theta$ on the lamination which is linearly expanded by the dynamics. 
That is, there exists $\lambda$ such that 
$$f^\star m_\theta = \lambda_\theta m_\theta$$
Here $h(\theta) = \log \lambda_\theta$.


Let $\theta$ be biaccessible in parameter space, i.e. so that there exists $\theta'$ such that $\theta \sim \theta'$. 
Then the Hubbard tree $T_\theta$ is finite, and the Julia set is locally connected.

Moreover, the piecewise linearly expanded metric on $\mathcal{L}_\theta$ semiconjugates $f_\theta$ on the Hubbard tree 
to a piecewise linear model of it, with slope $\lambda_\theta$. 

\subsection{Core entropy in higher degree}

\textbf{Question.} How many maxima does the entropy function have on the unicritical slice $f(z) = z^3 + c$?


